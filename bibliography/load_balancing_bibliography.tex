\documentclass{article}
\usepackage[style=numeric]{biblatex}
\usepackage[margin=1.0in]{geometry}


\addbibresource{biblatex-examples.bib}
\addbibresource{load_balancing_bibliography.bib}

%% declare this to include abstrats in the printed bibliography
%%
\DeclareFieldFormat{abstract}{\par\small#1}
\renewbibmacro*{finentry}{\printfield{abstract}\finentry}


\begin{document}

\title{Load balancing bibliography}
\author{Alan Heirich and Karthik Murthy}

\maketitle

\section{Categories}

\subsection{introduction}

\begin{itemize}
\item
Unbalanced tree search benchmark used by lifeline mapper
\cite{Saraswat:2011:LGL:1941553.1941582}.
Is there a standard benchmark for AMR simulations?

\item
Does the LB algorithm depend on the application?  Does it depend on the programming model (Legion)?  Is it independent of both?

\end{itemize}

\subsection{Category axes}

\begin{itemize}
\item
local vs. global: does the entire computer system have to participate, or can a small subset balance locally?

\item
nearest neighbor vs. long range exchanges

\item 
static vs. dynamic

\item
expensive vs. cheap (related to static vs. dynamic)

\item
embarassingly parallel vs. interconnected: does mapping matter?

\item
continuous domain (mesh simulations) vs. discrete (tree search)

\end{itemize}


\subsection{diffusion}

\let\clearpage\relax
%\begin{center}
%\begin{tabular}{ |c|c| } 
% \hline
% challenges:\\
- rebalance dynamically\\
- balancing based on entities\\
- diffusion with a local-exchange, i.e., bottom-up is bad\\
pros:\\
-log n approach since its top down\\
-communication cost accounted for, in a way\\
cons:\\
-a binary tree approach\\
-assume work load characterization\\
NOTE: Add [5] reference to the collection, its asurvey paper on load balancing\\
 & cell 1\\ 
% cell4 & \include{leiber} \\ 
% cell7 & cell8 \\ 
% \hline
%\end{tabular}
%\end{center}

\cite{Diamond:2017:DLB:3148226.3148236}\\

notes on \cite{Diamond:2017:DLB:3148226.3148236}
from Alan

finite element, finite volume unstructured adapting meshes

Diffusive partition improvement, application specified criteria

N-graph: hyper graph structure represents relations among elements

Diffusion is performed on the N-graph

A multigraph allows multiple edges between a pair of vertices

The N-graph does this for hyper graphs

Saves on memory versus just a graph

Imbalance = $T_max / T_mean$

Does not explain how to compute the transfer amount

Graph distance: migrate elements in order according to their distance from the cell center

Experiments : billion element mesh airplane tail structure

Argonne Mira blue gene Q

$128*2^{10}$ to $512*2^{10}$ elements cases

Showed reasonable improvement, 1.5 imbalance to 1.12

Did worse on larger problems



\cite{HORTON1993209}\\
challenges:\\
- rebalance dynamically\\
- balancing based on entities\\
- diffusion with a local-exchange, i.e., bottom-up is bad\\
pros:\\
-log n approach since its top down\\
-communication cost accounted for, in a way\\
cons:\\
-a binary tree approach\\
-assume work load characterization\\
NOTE: Add [5] reference to the collection, its asurvey paper on load balancing\\



notes on \cite{HORTON1993209}
from Alan

fea unstructured adaptive mesh

No topology assumptions

Multilevel algorithm complexity is logarithmic in number of processors

Diffusion methods may require many iterations, see Boillat claims of $O(n^2)$ iterations on n processors

Claim: pairwise diffusion can result in large load imbalance (I don’t think this is true of Laplace iteration although low frequency disturbances subside slowly)
See Cybenko claim of $log_2(n)$ steps iteration, uses hypercube topology

The algorithm here achieves $O(log_2 n)$ but does not depend on topology

Local communication costs less than nonlocal : hypercubes (this will always be true, just how much)

Load balancer should respect existing adjacency relationships of the domain

topology of the mesh may not match topology of computers

Basic diffusion method pairwise exchange of $0.5*(l_i - l_j)$ units of work

Multilevel algorithm aims to eliminate large scale imbalances

At each level divide processors into two sets and balance them as with two individuals

No explanation of how to choose these sets

Proof that it takes $log_2(n)$ steps — duh

requires entire system rebalance at once, not a local method

Claims standard diffusion techniques are bad because they don’t guarantee number of iterations




\cite{Deng:2010:HDB:1889863.1889910}\\

notes on
\cite{Deng:2010:HDB:1889863.1889910}
from Alan

Efficient cell selection scheme

Local and global diffusion schemes, global performs best

Global means knows all servers, local means only knows nearest neighbors

Note: reference ou and ranka 1997 solve lb problem as linear programming

Distributed virtual environments prefer fast solution over optimal solution

Experiment using simulated workloads, virtual environment users moving through the environment, environment is partitioned into regions, one region per server


\cite{Lieber:2016:PDL:2966884.2966887}\\
challenges:\\
-diffusion vs geometrical vs graph-based methods\\
-tasks are migrated based on geometrical coordinates or graph topolgogy\\
-geometrical good balance but lot of migration, e.g., gossip\\
-diffusion never really considered in HPC\\
this is a survey paper[a very good reference]:\\
- load transfer vector prepared by node to move tasks to another node\\
- CHEBY algorithm not uses difference in work load\\
- PARMETIS uses diffusion, ZOLTAN library and GLB library should be in mind\\
- chemotaxis uses capacity on me - load on target\\
- which tasks to send based on comm reduction, i.e., comm with neighbors only\\
their questions are our questions:\\
-how can we apply diffusion on today’s common hardware topologies like fat trees?\\
-What is a fast and high-quality method for task selection allowing to trade-off balance, migration and edge-cut? \\
-How do we scalably implement the termination criterion for diffusion when no fast collectives are available? E. g. use approx or auto-tune iteration count?\\ 


notes on
\cite{Lieber:2016:PDL:2966884.2966887}
from Alan


Good survey paper, worth reading again

Compare diffusion to geometric and graph based methods on thousands of nodes

Space filling curves, recursive bisection, parMetis, hierarchical space filling curves

Concludes diffusion has advantages

	.	The second-order algo- rithm [15] extends OD such that the previous iteration’s transfer influences the current. The parameter β ∈ ( 0,2) controls the influence. Optimal values are derived in [9] 

	.	[9]  R. Elsa ̈sser, B. Monien, and R. Preis. Diffusion Schemes for Load Balancing on Heterogeneous Networks. Theory Comput. Sys., 35(3):305–320, 2002. 

	.	[15]  S. Muthukrishnan, B. Ghosh, and M. H. Schultz. First 
and Second Order Diffusive Methods for Rapid, Coarse, Distributed Load Balancing. Theory Comput. Sys., 31:331–354, 1998. 

Survey based on improving diffusion

Original diffusion

Second order diffusion

Improved diffusion called cheby

Chemotaxis-inspired diffusion, additional round of exchange of capacity of the target node guides the diffusion locally

Dimension exchange - Xu and Lau

\cite{ROTARU2004481}\\

notes on
\cite{ROTARU2004481}
from Alan


Our contribution  can be summarized as follows: we give a direct explicit expression of the balancing flow generated by a generalized diffusion algorithm and we show that this flow has an interesting property, that it is a scaled projection of any other balancing flow in the same heterogeneous environment. We give estimations for the second largest eigenvalue of a generalized diffusion matrix and we estimate the complexity of the proposed algorithm.  We further show that this algorithm has a better convergence factor than the hydrodynamic algorithm [17,18]. Compared to other approaches, the one we consider here offers the advantage of not using parameters that are dependent upon the eigenvalues of the Laplacian of the communication graph. 

Solves load balancing and mapping

Since communication changes so frequently cannot afford to compute Laplacian eigenvalues

Analogy to markov chains

Connections between generalized diffusion matrices and Laplacian spectrum of the graph

Bounds on eigenvalues

Migration flow - expression for the flow

Good paper lots of analysis


Estimations were given for the maximum number of steps that such an iterative process may take to balance

purpose, some general bounds were formulated for the second largest eigenvalue of a generalized diffusion matrix. These bounds were also used to show that there are generalized diffusion algorithms that theoretically converge faster than the hydrodynamic algorithm 


\cite{ParabolicLB}\\
\cite{CYBENKO1989279}\\
\cite{10.2307/2584287}\\

notes on
\cite{10.2307/2584287}
from Alan

uses successive over relaxation to find an optimal step size for convergence

I wonder: do these convergence issues really matter?  Is the simplest scheme good enough?





\cite{Boillat:1990:LBP:95324.95326}\\

notes on
\cite{Boillat:1990:LBP:95324.95326}
from Alan

Show polynomial time convergence to equilibrium

Remark 5. In the discrete case, i.e. working with individual processes, our problem is equivalent to the random walk problem in graphs[20] 
20. D. Aldous. ‘An inaoduction to covering problems for random walks on graphs’.J. Theoretical 
Probability, 2(1), 87-89 (1989).


\cite{XU199572}
challenges:\\
-mapping + load balancing at the same time\\
-a form of diffusion is the answer for both\\
-goals of mapping, workload balance and comm time\\
pros:\\
-based on the laplacian matrix of communicating processes\\ 
-delay and fault tolerant, insensitive to problem scale, convergence depends on initial conditions (hmm)\\
cons:\\
-np-complete problem\\
-m to be 2, planar connection networks, but arbitrary topologies like hierarchical, maybe linearization, not sure: did not understand the mapping to torus by a broadcast-type strategy, what about binomial dissemination?.\\
-reduction of the problem for a specific instance ? isnt it ?





\cite{SCHLOEGEL1997109}






\subsection{game theory}

\cite{GROSU20051022}




notes on
\cite{GROSU20051022}
from Alan


Static load balancing problem

Noncooperative game: processors work independently to arrive at equilibrium

Characterize Nash equilibrium and derive greedy algorithm to compute it

Assume Poisson arrivals, exponentially distributed task times

$\phi_i$ job generating rate at node I

$s_{ji}$ fraction of user j tasks to be sent to node i

$\mu_i$ processing rate at node i 

Load balancing strategy for user j is a vector of ${ s_{ji} }$

Minimize response time for user j

Remark: this is for multiple users (humans) submitting jobs to a distributed system (cluster).

Our case is one user (application) submitting tasks to an exascale system

“Best reply” for a user is a strategy that gives minimal response time for that user in light of other users strategies

Similar problem for one user treated in Tang and Chanson[35]
X. Tang, S.T. Chanson, Optimizing static job scheduling in a network
of heterogeneous computers, in: Proceedings of the International
Conference on Parallel Processing, August 2000, 373–382. Not very interesting.

Equation (8) defines the BEST\_REPLY solution

Execution time is O(n log n) due to the need to sort computers by workload

Otherwise it would be O(n)

This algorithm requires global knowledge of the workload of every computer at every agent and knowledge of all users strategies

Not scalable

Compared to two other lb schemes, one does a global optimization based on global knowledge, and an IOS scheme that gives good quality results

Experiments on 16 node cluster


\cite{doi:10.1142/S0219198902000574}

notes on
\cite{doi:10.1142/S0219198902000574}
from Alan

Establish uniqueness of Nash equilibria

No software experiment

Not relevant






\cite{7967109}

notes on
\cite{7967109}
from Alan

Entirely theoretical result, random movement of tasks with uniform weights



\cite{BELIKOVETSKY201616}

notes on
\cite{BELIKOVETSKY201616}
from alan

Load Rebalancing Games in Dynamic Systems with Migration Costs 
Initial assignment of jobs to identical parallel machines

Machines are added or subtracted

Extension parameter $\sigma$ is added to the cost of a job after it is moved (this is intended for a one time cost, not repeated)

Paper proves existence and calculation of Nash equilibrium and Strong Nash equilibrium

Under some assumptions any stable modified schedule approximates well an optimal schedule

“Each job incurs a cost which is equal to the total load on the machine it is assigned to” … ????

See game theoretic treatments of this problem 12, 2, 5, 8, survey in 17
	.	[2]  N. Andelman, M. Feldman, and Y. Mansour. Strong Price of Anarchy. In SODA, 2007 

	.	[5]  A. Czumaj and B. V ̈ocking. Tight bounds for worst-case equilibria. In ACM Transactions on Algo- 
rithms, vol.3(1), 2007 

	.	[8]  A. Fiat, H. Kaplan, M. Levi, and S. Olonetsky. Strong Price of Anarchy for Machine Load Balancing. In ICALP, 2007. 

	.	[12] R.L. Graham. Bounds on Multiprocessing Timing Anomalies. SIAM J. Appl. Math., 17:263–269, 1969. 

	.	[17] B. V ̈ocking. In N. Nisan, T. Roughgarden, E. Tardos and V. Vazirani, eds., Algorithmic Game Theory. Chapter 20: Selfish Load Balancing. Cambridge University Press, 2007. 

$s_0$ assignment of $n$ jobs on $m_0$ machines
$m’$ machine are added or removed

Seek a Nash equilibrium $s$ where no machine can improve its situation by changing machines

``Assume that some preprocessing is done at the time a client is assigned to a server, before the download actually begins (e.g., locating the required file, format conversion, etc.). Clients might choose to switch to a mirror server. Such a change would require repeating the preprocessing work on the new server.   Another example of a system in which extension penalty occurs is of an RPC (Remote Proce- dure Call) service. In this service, a cloud of servers enables service to simultaneous users. When the system is upgraded, more virtual servers are added. Users might switch to the new servers and get a better service (with less congestion), however, some set-up time and configuration tuning is required for each new user.  Note that in all the above applications, the delay caused due to a migration is independent of the migrating job. 
’’

These sound like one-time costs to me hence my objection to this analysis, also the cost may depend on the job (amount of data movement, size of code, number of open files, etc)

Section 1.1 some issues with notation, use of $L_i(s)$ and $L_s(j)$ to mean different things?
“Price of anarchy” = ratio between max cost of a Nash equilibrium and the optimum schedule (load imbalance)
“Price of stability” = ratio between min cost of a Nash equilibrium and the optimum schedule

A set of players form a “coalition” if each job moves and strictly reduces its cost
Assignment is a “strong Nash equilibrium” if there exists no such coalition
Similar notions of strong price of anarchy, strong price of stability

``The simple greedy List-scheduling (LS) algorithm [11] provides a $(2 - 1/m)$approximation to the minimum makespan problem. A bit better approximation ratio of  $((4/3) - (1/3m))$ is guaranteed by the Longest Processing Time (LPT) algorithm [12]. A PTAS for the minimum makespan problem on identical machines is given in [13]. ‘’

``We show that any job scheduling game with added or removed machines possesses at least one Nash equilibrium schedule. Moreover, some optimal solution is also a Nash equilibrium, and thus, the price of stability is 1. We show that in general, the price of anarchy is unbounded when machines are either added or removed. ‘’

``We note that in a dynamic setting in which machines are added or removed and migrations 
are free of cost (i.e., when $\sigma$ = 0), then the results known for classic load balancing games apply.   In particular, the P oA assuming $\sigma$ = 0 is $2 - (2 / (m+1))$ for a game with m machines in the modified $m+1$  systems. The proofs are identical to the proofs for a fixed number of machines. Thus, the difference between our results and the results for the classic load balancing game are due to the migration penalty. ‘’

Sections on machine addition, machine removed, coalitions.  Each case proves existence of solutions.




\cite{BARAM2014241}


Reoptimization of the minimum total flow-time scheduling problem 

Guy Baram, Tami Tamir

Dynamic load balancing, accounts for migration cost
Optimal algorithm to find optimal solution with minimal migration cost (transition cost)
Minimal solution can be found by greedy algorithms, Shorted Processing Time(SPT) assign jobs in nondecreasing order by length
Example applications: manufacturing systems with jobs migrated along production lines; live migration of running VMs in a cloud data center; RPC servers

Other papers that minimize the migration cost:

	.	[10]  C. Clark, K. Fraser, S. Hand, J.G. Hansen, E. Jul, C. Limpach, I. Pratt, A. Warfield, Live migration of virtual machines, in: The 2nd Symposium on Networked Systems Design and Implementation (NSDI), 2005. 

	.	[19]  S. Hacking, B. Hudzia, Improving the live migration process of large enterprise 
applications, in: The 3rd International Workshop on Virtualization Technolo- 
gies in Distributed Computing (VTDC), 2009. 


J0 set of n0 jobs
M0 set of m0 identical machines
Pj is processing time for job j
S0 schedule of the initial instance
Changes include adding/deleting jobs from J0, machines from M0, and changing values of pj

A machine can process at most one job at a time
Price list theta(i,i’,j) is cost to migrate job j from machine I to machine I’
Job extension penalty delta(i,i’,j) is time extension increase in pj after moving job j from machine I to I’
Cj completion time of job j
Minimize sum Cj = “total flow time”

Two questions:
1. Reschedule to find minimum possible transition cost
2. Reschedule within a given budget B

Section 2, time t=0: optimal algorithm for rescheduling, complexity based on complete matching in bipartite graph with O(nm) vertices
Section 3, time t>0:
Section 4: unit migration costs and no job extension penalty.
Section 5: rebalancing with a budget
Section 6: results and discussion

Section 2 ——————————

	.	[7]  J.L. Bruno, E.G. Coffman, R. Sethi, Scheduling independent tasks to reduce mean finishing time, Commun. ACM 17 (1974) 382–387.

	.	[18]  W. Horn, Minimizing average flow-time with parallel machines, Oper. Res. 21 
(1973) 846–847. 

Problem instance: nxm matrix of job times, pij is time for job j on machine I
Seems like a bug in the paper with indexing here, I is jobs and j is machines

Construct a bipartite graph, nodes V = J u U
J is the set of n jobs
U is the set of mn nodes

“Node qik is the kth from last position on machine I”
Edges completely connect nodes in J to nodes in U
Edge weights are w’(vj, qik) = k p’ij
“If job j is the kth from the last job to run on machine I, it contributes k times p’ij to the sum of completion times”

?? Why k times?

Optimal solution is found by min-weight matching in the bipartite graph.  (Obvious)

*** this algorithm is not distributed, requires global knowledge and centralized computation, ergo not scalable












\subsection{mapping}

\cite{doi:10.1137/0611030}


Heuristic algorithm: compute Fiedler vector, use it as an edge separator, then find the vertex separator by maximum matching in a subgraph

Require the two parts of the graph to be “roughly equal” in order to achieve load balance (not exactly equal)

Number of zero eigenvalue of laplacian matrix is equal to the number of connected components of the graph

Lots of definitions of Fiedler vector, laplacian matrix, spectral theory

4. Partition of grid graphs

Start with a “path graph” ?

Laplacian matrix is tridiagonal -> path graph is x 0 x diagonal pattern

5 spectral partitioning algorithm

Compute Fiedler vector

Use median value of vector elements to partition vector into left, right halves
This is an edge separator

We require a vertex separator
Simplest method is to choose the smaller of the two  endpoint sets (vertices divided by the edge separator)

But we want to compute the smallest separator
Solution is a vertex cover of the bipartite graph
Remove vertices a,b such that every edge is incident on one or both of these vertices

To find a minimum vertex cover use maximum matching algorithm

Worst case time complexity for algorithm to find one partitioning is O(sort ne)
Where e is number of edges, n is the length of the Fiedler vector



\cite{4227986}


Hyper graph partitioning for dynamic lb.  minimizes sum of communication costs plus migration cost.

Parallel multilevel repartitioning with Zoltan load-balancing toolkit

Results quality compressed favorably to ParMETIS

$t_total = alpha ( t_compute + t_communication ) + t_migration + t_repartitioning$

They ignore $t_compute$ and $t_repartuition$: for $t_compute$ they assume a balanced workload; for $t_repartition$ they assume this time is negligible

Net = hyper edge (edge with more than two vertices)

Vertices have weights wi 

Nets have costs cj (like edge weights)

Partition the hyper graph - each partition corresponds to one computer

Hypergrah H, Partition P

Wp = sum of vertex weights in partition Vp

P={Vi}. Partition consists of a set of parts 

Connectivity $\lambda_j$ of net j is the number of parts it connects

Multilevel partitioning - fine to coarse strategy, partition the coarse graph project back up

Their model combines communication cost with migration cost

Epoch = execution in time between two dynamic rebalance operations

Repartitioning hyper graph - augment the current epoch hyper graph Hj with additional vertices and nets to model data migration costs.  Now partition the resulting hyper graph using fixed vertices.

$H^j = (V^j, E^j)$ is the hyper graph that models epoch j of the computation

Rebalance following each epoch

Construction of the repartitioning hyper graph:
Add k new partition vertices ui, with zero weight, for each of k partition
For each vertex v in the old hyper graph, add a ”migration net” from v to new vertex ui if v was in partition I

Figure 1 is confusing read the text carefully

Vertex ui must be fixed to partition I

Now partition the new hypergraph

If the partition cuts a migration net then the cost of migration gets counted in the overall communication cost
This is the key idea that accounts for migration cost

Parallel multilevel hypergdaph partitioning with fixed vertices

Hg partition is NP hard, approximate using multilevel heuristics

Like multigrid,  coarsen the graph and partition the coarsest level

Propagate the partitions upward to the full graph


\cite{doi:10.1142/S0129054197000215}


\cite{Sbirlea:2014:BMS:2628071.2628090}


Concerned with scheduling when memory is limited.

“For many parallel applications, the memory requirements can be significantly larger than for their se- quential counterparts and, more importantly, their memory utilization depends critically on the schedule used when run- ning them. “

“Using the inspector/executor model, BMS tailors the set of allowable schedules to either guar- antee that the program can be executed within the given memory bound, or throw an error during the inspector phase without running the computation if no feasible schedule can be found. “

“Since solving BMS is NP-hard, we propose an approach in which we first use our heuristic algorithm, and if it fails we fall back on a more expensive optimal approach which is sped up by the best-effort result of the heuristic. “
“Unfortunately, parallel execution is known to increase mem- ory requirements compared to a serial baseline [8,11]. “	.	

[8] Robert D. Blumofe and Charles E. Leiserson. “Scheduling multithreaded computations by work stealing”. In: J. ACM (1999). 

[11] F. Warren Burton. “Guaranteeing Good Memory Bounds for Parallel Programs”. In: IEEE Trans. Softw. Eng. (1996) “

[16] 	.I. Dooley et al. “A study of memory-aware scheduling in message driven parallel programs”. In: HiPC. 2010 


This problem is a general case of the register sufficiency problem (NP hardness comes from this)

Inspector/executor: inspector builds a task graph that shows parent-child task relationships and reader-writer relationship for the data (similar to Legion).
Inspector identifies scheduling restrictions that lead to bounded-memory execution.  Enforce these restrictions in the executor stage when the app runs a load balancing work-stealing scheduler.

Contributions:


Heuristic algorithms for bounded memory scheduling (BMS) based on inspector/executor
Optimal algorithm based on ILP

Schedule memoization for heuristic algorithm

Concurrent Collections Programming Model (CnC)

Tasks (called steps)

Step tag - identify specific instance of a step (task)

items - dynamic single assignment variables

Item collections - identified by key, collect related items together like a struct

Step reads items by calling item\_collection.get(key), blocks waiting for data
They claim that building the task graph dynamically is too late for bounding memory consumption (?)

“Many analyses of task-parallel programs (such as data race detection) require understanding the task-parallel struc- ture of the computation, which is usually unknown at com- pile time. As a result, many of these analyses build the task graph dynamically, while the application is running. Unfor- tunately, this is too late for certain optimizations, such as bounding the memory consumption of the program. “

“Expansion functions” augment the CnC task graph to provide the necessary information

*******
“If the keys of items read and written and tags of steps spawned are only a function of the current step tag, then the application has independent control and data, which is needed to accurately model an application using BMS. “ *****
Sounds terrible!

“If the keys and tags depend on the values of items, we say that the application has coupled control and data. 
When faced with an application with coupled control and data, one possible solution is to include more of the com- putation itself in the graph expansion functions. In the extreme case, by including all the computation in the ex- pansion functions, we would be able to obtain an accurate dynamic task graph. Unfortunately, in the worst case, the computation would be performed twice, once for the expan- sion and once for the actual execution. However, our expe- rience is that many application contain independent control and data, thereby supporting the BMS approach. “

Assume all tasks have unit size, relax this in section 8

BMS heuristic algorithm is based on list scheduling

If no acceptable solution is found, run the optimal algorithm

5.1 successive relaxation of schedules
Sort depth-first rather than breadth-first in the task graph (duh)

5.2 color assignment
Determines concurrent chains of tasks

6. Optimal algorithm through ILP

8.1 supporting multiple item sizes
Assigns memory locations, deal with fragmentation that results from different sizes




\subsection{heterogeneous}

\cite{Flegar:2017:OLI:3149704.3149767}



Overcoming Load Imbalance for Irregular Sparse Matrices 
Flegar and anzt. $IA^3$ ’17 : Irregular Applications Architecture and Algorithms

GPU implementation of sparse matrix-vector product
Uses COO format, row-column index of each element
ELL format - store only nonzero, pad with zeros to give same vector length, for SIMD

This is not a load balancing or mapping algorithm.  This is a matrix multiply algorithm that tries to be load balanced.


\bigskip

\cite{8082085}

\bigskip
Dynamic Load Balancing on Multi-GPUs System for Big Data Processing 

This paper presents a novel dynamic load balancing model for heterogeneous multi-GPU systems based on the fuzzy neural network (FNN) framework. The devised model has been implemented and demonstrated in a case study for improving the computational performance of a two dimensional (2D) discrete wavelet transform (DWT). 

Chen et al. [7] proposed a task-based dynamic load balancing solution for multi-GPU systems that can achieve a near-linear speedup with the increase number of GPU nodes. Acosta et al. [8] had developed a dynamic load balancing functional library that aims to balancing the load on each node according to the corresponding system runtime. However, these pilot studies are based on the assumptions that all GPU nodes equipped in a multi- GPU platform have equal computational capacity. In addition, the task-based load balancing schedulers these approaches relied upon fall short to support applications with huge data throughputs but limited processing function(s) since there are very few “tasks” to schedule, e.g. DWT. 

	.	[7]  L.Chen,O.Villa,S.Krishnamoorthy,andG.R. Gao, “Dynamic Load Balancing on Single- and Multi-GPU Systems,” Ipdps, 2010. 


	.	[8]  A.Acosta,R.Corujo,V.Blanco,F.Almeida,H.P. C. Group, and E. T. S. De Ingenier, “Dynamic Load Balancing on Heterogeneous Multicore / MultiGPU Systems,” 2010. 


Their algorithm: divide work into equal size units; dynamically schedule those units according to evolution of dynamic workload; use fuzzy neural network to do this (?)

Fuzzy neural network predicts execution time for each GPU based on: flops rate; memory size; parallel scaling; occupancy rate of compute resources; occupancy rate of global memory.
Train neural network based on historical records.  (Main advantage seems to be that it allows GPUs to have different capacities).


CASE STUDY - discrete wavelet transform
Same amount of work per work-item.
Study used two different GPUs of different capacities.
Only experimented with two GPUs.

I do not think this paper is a good idea.


\cite{7993387}

A performance, power, and energy efficiency analysis of load balancing techniques for GPUs 

Regular and irregular workloads, hard to give good performance in both cases.
Dynamic multi-phase workload partition and work item-to-thread allocation.
Low complexity, good results.
Paper compares this to several other state of the art GPU lb algorithms.


NV Maxwell GTX 980, NV Jetson Kepler TK1 (low power embedded).
Work-units are grouped into work-items.
BFS graphs: work-units are graph vertices; work-items are neighbors of each work-item.
Prefix-sum-array holds offset to each work-item.


PRIOR ART:

Static mapping:


[4] P. Harish and P. J. Narayanan, “Accelerating large graph algorithms on the GPU using CUDA,” in Proceedings of the 14th International Conference on High Performance Computing, ser. HiPC’07, 2007, pp. 197–208. 
Simplest, one work item per thread, not load balanced

[5] S.Hong,S.K.Kim,T.Oguntebi,andK.Olukotun,“AcceleratingCUDA graph algorithms at maximum warp,” in Proceedings of the 16th ACM Symposium on Principles and Practice of Parallel Programming, ser. PPoPP ’11, 2011, pp. 267–276. 
Virtual warp (group of threads): workload assigned to threads of same group almost equal, therefore reduce branch divergence and improve memory coalescence.
Have to size the virtual warp correctly to get performance, this is a static parameter.
Good lb within a warp but not balanced across warps.

Semi-dynamic mapping:

[6] F. Busato and N. Bombieri, “BFS-4K: an efficient implementation of BFS for kepler GPU architectures,” IEEE Transactions on Parallel Distributed Systems, vol. 26, no. 7, pp. 1826–1838, 2015. 
Dynamic virtual warps: VW calculated at runtime, lb within a warp and across warps.

[7] D. Merrill, M. Garland, and A. Grimshaw, “Scalable GPU graph traversal,” in Proceedings of the 17th ACM SIGPLAN Symposium on Principles and Practice of Parallel Programming, ser. PPoPP ’12, 2012, pp. 117–128. 
Perfect balance among threads and warps, but expensive strip-mine first step, not worth it for regular workloads, good for irregular.

Dynamic mapping:

All require binary search across prefix-sum array

[8] A. Davidson, S. Baxter, M. Garland, and J. D. Owens, “Work-efficient parallel gpu methods for single-source shortest paths,” in Parallel and Distributed Processing Symposium, 2014 IEEE 28th International. IEEE, 2014, pp. 349–359. 
Load prefix-sum chunks to GPU shared memory, process on GPU, do not guarantee balance across blocks, nor memory coalescing.

[9] O. Green, R. McColl, and D. A. Bader, “Gpu merge path: a gpu merging algorithm,” in Proceedings of the 26th ACM international conference on Supercomputing. ACM, 2012, pp. 331–340. 
[10] “Modern gpu library.” [Online]. Available: http://nvlabs.github.io/ moderngpu/ 
9 and 10 propose similar methods, two phases: partition and expand.
First partition prefix-sum array into balanced chunks.
Second expand all threads load their chunks into shared memory, each thread binary searches the prefix-sum array to get the first assigned work unit.
Balanced across threads, warps and blocks at the cost of two binary searches.
Problem: memory accesses of the threads to work-units is badly scattered in memory.

[12] F. Busato and N. Bombieri, “A dynamic approach for workload partitioning on GPU archi- tectures,” IEEE Trans. on Parallel Distributed Systems, vol. preprint, no. 99, pp. 1–15, 2016. 
Multiphase: like two phase but at lower cost.

A. Multi-Phase Technique

Hybrid partitioning phase: each thread searches work-items switching between optimized binary search and interpolation search.
Iterative coalesced expansion phase: all threads load chunks into shared memory; each thread does optimized binary search to get the work-unit; three iterative sub-phases reorganize memory access for better coalescence.

1 writing on registers

2 shared memory flush, data reorganization

3 coalesced memory access

Steps 2 and 3 repeat until all are processed.  Results perform better due to improved memory coalescence.

Results (lots of data):
Their method how lowest execution time and lowest energy use out of 12 methods.  Next best was virtual warps.


\cite{Cederman:2008:DLB:1413957.1413967}


2008 graphics hardware
On dynamic load balancing on graphics processors

Task weight not known in advance, new tasks created dynamically during execution.
Compare 4 dlb methods, 1 is lock based.
Test problem is octree partition of particles.
Result: synchronization is very expensive, lock-free is better.

4 dlb methods:
Centralized blocking task queue (lock based),
Centralized non blocking task queue,
Centralized static task list,
Task stealing.

Thread level scheduling.
Gpu thread block: equal size, all on one processor, fast local memory, barrier.
Warp: 32 consecutive threads.
Very old processors: 9600GT 512MB 64 cores.

Task stealing performed best.


\cite{10.1007/978-981-10-6442-5_56}

A Load Balancing Strategy for Monte Carlo Method in PageRank Problem.
PageRank can be solved by Monte Carlo (known result).
Implement PageRank on GPU, deal with instruction divergence.
“Adopt the low-discrepancy sequences to simulate the random walks in PageRank computations”
“Each thread of a block to compute a random walk of each vertex with a same low-discrepancy sequence”

PageRank: list the relevant pages in order.
It’s a stationary vector of a random walk that simulates the process of surfing the internet.
Interpret as frequency that a random surfer browses the web page, similar to popularity.

Connectivity matrix P defines all hyperlinks.
P-ij = 1/k if page has k outgoing links and j is one of the links,
 = 1/n if page has no outgoing links,
Else = 0.
Surfer will choose next page from one of the links at random with p=c otherwise randomly from the web with p=(1-c).
So this is a Markov process with transition matrix

PP = c P + (1 - c) (1/n) E

Damping factor c = 0.85,
E is matrix of all ones.
The PageRank is the stationary distribution of the Markov chain, row vector pi such that
pi PP = pi , pi 1 = 1

Linear algebra methods compute pi using power method iteration, pi := pi PP, approximately linear in n.
Monte Carlo methods are faster and highly parallel.
This paper considers efficient Monte Carlo on GPU.

Optimization issues:

Instruction divergence

Load balancing between threads of a warp

Memory access conflicts between threads

Cache performance among threads

Local memory utilization

Instruction divergence in Monte Carlo occurs because random numbers are different so Markov chains diverge.
So they replace random sequences with “low discrepancy” sequences that are repeatable and identical, they reduce the variance of Monte Carlo Sampling [9-11]

9. Cervellera, C., Macciò, D.: Low-discrepancy points for deterministic assignment of hidden 
weights in extreme learning machines. IEEE Trans. Neural Netw. Learn. Syst. 27(4), 891– 896 (2016) 

	10.	Gan, G., Valdez, E.A.: An empirical comparison of some experimental designs for the valuation of large variable annuity portfolios. Dependence Model. 4(1) (2016) 

	11.	Zapotecas-Martínez, S., Aguirre, H.E., Tanaka, K., et al.: On the low-discrepancy sequences and their use in MOEA/D for high-dimensional objective spaces. In: 2015 IEEE Congress on Evolutionary Computation (CEC), pp. 2835–2842. IEEE (2015) 

We propose a divergence avoidable strategy to allocate the threads of a block to compute a random walk of each vertex with identical low-discrepancy sequence. Hence, the threads of a block will simultaneously execute same instruction to compute next state of Markov chain or terminate. We can address the instruction divergence by our strategy. 

The static load balancing strategy pre-computes the length of each Markov chain to reduce the cost of absorbing state determination. The dynamic load balancing strategy dynamically determines the length of Markov chains to avoid the cost of loop condition determination. We will use experiments to test the efficiency of two strategies. 

Another optimization is that we load the low-discrepancy sequences into shared memory to speed up data fetch operations. 

The experiments indicate the bottleneck of our strategy is the memory access conflicts between threads. 
we store the adjacency matrix of a graph as the Compressed Sparse Row (CSR) format 
Since each random walk can be simulated independently, we can easily assign the computation of each walk for each thread. Furthermore, the all random walks begin from same node will be assigned to one block and all threads in a block simulate all random walks starting from one vertex in parallel. 
the time performance of a block will be critically influenced by the thread with longest random walks. 
So we try to use the determinacy of quasi random numbers to solve this problem.

Static load balancing

Wk, the expectation number of random walks whose length is k.
we can call T kernels which have constant number of threads, Wk, each kernel computes k steps so that random walks in block have same length, in other words under same workload.   
all threads in a block have same workload, as a result, the warp execution efficiency will be improved. 
Their experiments show the dynamic lb made a minor improvement to the static lb case.

This is a good paper.





\cite{dlbgraphgpu}


Dynamic Load Balancing Strategies for Graph Applications on GPUs 

Existing work: node-based work-assignment to threads, gives poor load balance; edge-based requires lots of memory.
New work: three improvements to ameliorate these problems

Breadth-first search BFS;
Single source shortest past SSSP;
Minimum spanning tree MST;
“Betweenness centrality”;
Graph500 benchmark database.

Node-based assignment leads to load imbalance if the degree of the nodes is not uniform, works well for data in Compressed Sparse-Row (CSR) format.

Edge-based don’t have this problem but require reformatting the data to Coordinate List (COO).  Takes too much storage to handle large graphs on limited GPU memory.
Edge based requires distributivity in the operator which is not always possible,

Three contributions:
Workload decomposition - assign edges, but only for nodes on the active list
Node splitting - split a high degree node into multiple low degree nodes
Hierarchical processing - hierarchy of work lists

Evaluated on BFS and SSSP algorithms

Workload decomposition:

In this approach, the processing elements in the worklist continue to be the nodes, but the workload of the nodes, namely, the edges, are decomposed across threads using a block distribution. E number of graph edges are partitioned across T threads such that each thread receives a contiguous chunk of E/T edges for processing. Thus, a given thread processes a subset of edges corresponding to a subset of nodes and all the edges outgoing from a node may not be processed by the same thread. 

An advantage of workload decomposition is that it works with the CSR format and therefore, has a lower space com- plexity. 
A drawback of the workload decomposition is that it can lead to uncoalesced accesses since a node’s edges may get separated. 
In our experiments, we observe that the limitations of work- load decomposition affect its performance for large-diameter graphs (such as the road networks) but the method performs very well for scale-free graphs such as the social network 

Node splitting

node splitting preprocesses the graph to split each high-degree node into multiple low-degree child-nodes.
node-splitting approach has the advantage that it can work with the space-efficient CSR representation. 
A salient feature of our node splitting strategy is to automatically determine the threshold MDT for node splitting. Obvious methods based on a threshold or max-degree etc. do not work in general.   we use a histogram based method in which we use HistogramBinCount number of bins representing the ranges of out-degrees of the nodes in the original graph. 

An advantage of the node splitting approach is that it con- tinues to work with the space-efficient CSR format.
all the edges of a node are processed by the same thread, reducing bookkeeping and improving the scope for memory coalescing. The primary disadvantage of node splitting is that it results in extra atomic operations to update the child nodes whenever the parent node gets updated. A secondary disadvantage is the overhead of computing the histogram to find the MDT. 

node- splitting provides considerably better load-balancing. In ad- dition, it provides comparable performance for large diameter graphs (such as road networks); but it has a high overhead for power-law degree distribution graphs. 

Hierarchical processing

Hiearchical processing performs a time-decomposition of the workload. It achieves this by partitioning the main (super) worklist into several sub-worklists. If the sub-worklist is large, it can be further partitioned into sub-sub-worklists, and so on. This builds a hierarchy of worklists. The depth of this hierar- chy is tunable, and we utilize the histogram-based approach in node-splitting (Section III-B) for finding the maximum degree threshold (MDT) which determines when to split a worklist into sub-worklists. 

Compared BFS and SSSP to implementing from LonestarGPU which uses node-based distribution.


\subsection{large scale}

\cite{PEARCE2017}


Notes on Exploring dynamic load imbalance solutions with the CoMD proxy application 

CoMD - exascale codesign center applications, molecular dynamics
Proxy application to allow load balancing study
Parameterize: work granularity; initial imbalance; dynamic imbalance
Study a “wide range of initial and dynamic load imbalance scenarios”

High/low initial imbalance
Fast/slow rate of change for dynamic imbalance

Related work
Charm++ developed AMPI adaptive MPI
Over decompose the simulation domain, then balance by moving virtual processors

Molecular dynamics with interaction cutoff radius
Use a regular 3D grid to partition the particles
Ghost cells, periodic boundary conditions
MPI implementation (others exist)

Initial imbalance: remove some atoms in a spherical region
Dynamic imbalance: change simulation to cause particles to drift
Overdecompose, many domains per processor, keep a domain graph

“Simple load balancing algorithm based on spatial sort”
Spatially sort domains using a Hilbert curve
Partition the curve between processors
Centralized lb algorithm

Results
Lb algorithm cost grows because it is centralized
Fastest simulations use 8x over decomposition plus load balancing

No real conclusions


\cite{BERLINSKA201814}


Comparing load-balancing algorithms for MapReduce under Zipfian data skews 
Joanna Berlinska, Maciej Drozdowski 

Map-reduce = mapping, shuffling, sorting, reducing.
Unequal distribution of keys causes imbalance.
They valuate four types of algorithms to balance work during last three phases.
Conclusion: hybrid methods are necessary.

Map produces (key1, value1) pairs.
A “cluster” is a set of pairs with the same key1.
Reduce produces (key2, value2) pairs.
“Data skew” = unequal distribution of cluster sizes.
Routing of work from mappers to reducers is driven by key1 values.

Non-uniform distribution of key1 results in unequal sizes of mapper output files, therefore imbalance in the shuffle phase.
Different frequencies of key1 result in unequal sorting and reducing times.

In this study the analysis cannot be performed by measurement there use math models and simulation.
Contributions:
Performance models for map-reduce; power-law (zipfian) distributions of keys are used.
Four algorithms proposed to mitigate skew, implemented in a simulator, evaluated.

M - number of mappers
R - number of reducers
Omega - space of key1
$k_i$ in Omega : ith key
$sigma_i$ - size of cluster of $k_i$

Map phase - process equal sized “splits”, write output files with $key_i$ clusters in same file
Partitioning function - chooses output file for $key_i$
Partition - set of keys for which partitioning function return the same value
R - number of partitions

$s_j$ - size in bytes of partition j (sum of $sigma_i$)
Partitions are disjoint, and cover all of Omega
Partitioning function is usually of the form “hash($key_i$) mod R”

At the end of the mapping phase each mapper holds R local output files.

V - total size of input data set
Each mapper processes V/m bytes and produces Vpsi/m output bytes.
All mappers finish at the same time.

Master assigns partitions to reducers.
They model communication bandwidth during the shuffle phase.
After reading the assigned partitions reducers start processing them.
Sort key-value pairs by key1, non-preemptive, not transferable
Process them in log-linear time
Store the results in the distributed file system

Assume keys are distributed according to a power law (see equation 1)
Distribution at z=0 is normal, at z=1 is Zipfian
Evidence fro analyzing logs shows power law distribution of keys in real Hadoop runs.

It is hard to build an optimal MapReduce schedule in advance, data skews mu be overcome online (dynamically).
Four algorithms are proposed: easy or hard key partitioning, cropss linear or log linear key sorting cost.
Easy partitioning - random assignment of keys to reducers,
Hard partitioning - sorted keys assigned to reducers in order.

4.1 reference distribution algorithm
Round robin assignment of keys to reducers.
No load balancing.
Equation (2) gives optimistic prediction of time for this algorithm.

4.2 static algorithm
Over decomposition, create any partitions and assign the to reducers to achieve load balance.
Requires np-hard bin packing problem solution, approximate with a heuristic.

4.3 multi-dynamic algorithm
Similar to work stealing among reducers.

4.4 mixed algorithm
Combines the static and dynamic algorithms.

4.5 divide keys algorithms
Partitioning function is created at runtime.
Divide one key cluster among multiple reducers.
Merge the results after.

5. Events simulation.
Simulated everything.  Too hard to instrument real code.

6.1 data skew
Best performance was the mixed algorithms with medium data skew.
With large data skew divide keys was necessary.

6.2 size of intermediate data
When size is large use divide-keys

6.3 reducer parameters

6.4 communication parameters
For large C mixed is better than divide-keys.

7. Summary and conclusions
The static algorithm adjusts the load distribution before the shuffle phase, using the idea of fine partitioning (a.k.a. overdecomposition). The multi-dynamic algorithm performs many simple load balancing operations in the reducing phase. The mixed algorithm processes a part of data like the static algorithm, and sends the chunks containing the remaining data on requests from idle reducers. Finally, the divide-keys algorithm computes a partitioning function adjusted to the input data distribution and allows processing a key cluster on several reducers. 
1. If data skew is small (z <= 0.4), the best choice is in most cases algorithm mixed(0.5).
2. When the skew is moderate (0.5 <= z <= 0.7), algorithm mixed(0.9) usually obtains the best results. 3. When the skew is big, the winning algorithm depends more on the system parameters. 
(a) If a red, gamma 1, C are small, and gamma, m, V are big, then algorithm divide-keys is the best. (b) In the opposite case it is better to use mixed(0.5). 


\cite{8017633}

Dynamic Load Balancing Based on Constrained K-D Tree Decomposition for Parallel Particle Tracing 

evenly redistribute particles across processes based on k-d tree decomposition. Each process is assigned with (1) a statically partitioned, axis-aligned data block that partially overlaps with neighboring blocks in other processes and (2) a dynamically determined k-d tree leaf node that bounds the active particles for computation; the bounds of the k-d tree nodes are constrained by the geometries of data blocks 
In the task- parallel methods, particles are statically distributed to parallel processes; each process has access to the whole data.  In the data- parallel methods, the flow data are statically partitioned into blocks and distributed to processes; particles exchange between processes to finish the tasks. 
In this design, each process is as- signed with (1) a statically partitioned, axis-aligned data block that partially overlaps with neighboring blocks in other processes and (2) a dynamically determined k-d tree leaf node that bounds the active particles for computation. 
In static data partitioning, we initially subdivide the domain into n non-overlapping, equal-sized, and axis-aligned blocks, where n equals the number of processes. We then expand the blocks to overlap with other blocks as much as possible, given the memory limit of the process. The expanded parts, called ghost layers, essentially maximize the overlaps between blocks under the memory limit and thus enable the k-d tree decomposition with con- straints. During run time, we periodically redistribute particles based on the constrained k-d tree decomposition to balance the workload. 
Task-parallelism does not work well for this problem.
In our study, we redesigned the parallel tree construction algorithm for constrained k-d trees. We further use the constrained k-d trees to balance the number of particles in parallel processes in order to achieve dynamic load balancing. 
We first partition the input data into non-overlapping, equal-sized, axis-aligned blocks and then expand the ghost layers of each block up to the memory limit of each process. The blocks are loaded into memory, and the particles are initialized in the corresponding blocks.  Each particle is assigned to the block whose  “core” region excluding ghost layers contains the particle. 

The computation stage is an iterative process that alternately exe- cutes the particle redistribution and particle tracing. In the particle redistribution phase, the parallel processes collectively exchange un- finished particles based on the constrained k-d tree decomposition. In the particle tracing phase, each parallel process independently traces its unfinished particles without communication. 
We conducted a comprehensive performance study with three applica- tions: tracing densely-seeded streamlines in thermal hydraulics simu- lation (Nek5000) data, querying source-destinations in GEOS-5 sim- ulation data, and computing FTLE of Hurricane Isabel data.  In our test, we execute 8 processes per compute node, and we use up to 8,192 processes on Blue-Gene/Q.
The baseline approach in the performance study is a data-parallel particle tracing implementation. The initialization stage of the baseline approach is the same as in Section 4. The computation stage alternates the independent particle tracing and the collective particle exchange phases 

Our method does not require any data pre- processing.   Our method does not rely on any flow feature analysis.   We make no assumptions about the initial parti- cle distribution.  Our method supports both static and time- varying data in both 2D and 3D meshes, because k-d trees can decom- pose spaces in arbitrary dimensions.  The data movement of our method is minimal.  The distributed k-d tree decomposition is self-consistent and decentralized. 
Our method does not guarantee optimal and perfect load balancing.  processes have different workloads even if they are assigned the same number of particles.  the constrained k-d tree decomposition does not guarantee an even distribution of particles.  the k-d tree could be imbalanced.   at worst, when the ghost layers just overlap, our method degenerates to the baseline method because the splitting planes will be in fixed positions. 






\cite{DEVINE2005133}

New challenges in dynamic load balancing(2005)
Devine et al

Lb not solved, new models needed for non-square, non-symmetric, and highly-connected systems.  Heterogeneous architecture require non-uniform computing, network and memory.  These capabilities should be delivered in toolkits (e.g. Zoltan)
RB=recursive bisection, SFC=space filling curve
Zoltan framework provides many different partitioning algorithms.

Geometric partitioning by graph connectivity: good for FEM.  But crash sim requires geometric locality rather than connectivity.  Particle methods are better with locality (there is no graph).

Zoltan supports 
RB (coordinate, intertial)
and
SFC.
Geometric methods are effective when locality is important and/or graph is not available.
They do not account for communication costs.

Crash simulations require point- and box-assignment of objects to partitions.  They have a novel way to do this, shows some benefits.

Multicriteria geometric partitioning:
multiple constraints versus multiple objectives.
In DLB time to solution is more important than solution quality.
Introduced a heuristic for bisection of vector-valued criteria.
Table shows results, not dramatically different from ParMetis results.

Hypergaph models:
High connectivity, heterogenous topology, non-symmetric matrices.
“Because edges in the graph model are non-directional, they imply symmetry in all relationships, making them appropriate only for problems represented by square, symmetric matrices. Non-symmetric systems A must be represented by a symmetrized model $A + A^T $”
I do not agree, graphs can be directional.

Hypergraph models do not imply symmetry.
Each edge contains two or more related vertices, not just two vertices.
Serial hyper graph partitioning supplied by Metis but parallel partitioners are needed.
Example 3: hyper graph.
Only small advantage for FEM graph.
Showed better advantage in partition a polymer self-assembly simulation (Tramonto) with complex sparsity pattern.

Resource-aware load balancing for heterogeneous architectures.
DRUM=Dynamic Resource Utilization Model, provides system characteristics.
Tree structure of all processors, compute at the nodes.
Experiment using PHAML=parallel hierarchical adaptive multilevel software.
Showed better results than uniform partitioning.


\cite{javataskpool}



\cite{barat:tel-01672546}


Load balancing of multiphysics simulations by multi-criteria graph partitioning
(Thesis, university of Bordeaux)

Multiphysics with several computational phases.
For each phase, balance the workload and minimize the communication.
Problem is NP-hard, this thesis explores heuristic solutions.
Suggest improvements to “the multilevel algorithm”.
Two algorithms based on vertex movement: move any vertex that improves the solution; move the vertex that makes the largest improvement to the solution.
Results: their methods returned balanced partitions where MeTiS fails regularly.

Mesh based simulations.
Vertex weight indicates workload.
Multiphysics simulations has a vector of weights at each vertex.

Multilevel algorithm = coarsening phase, partitioning phase, refinement phase.
They will show improved heuristics for coarsening and refining, and explore various forms of partitioning.

S - set.
P(S) - set of all subsets of S.
Pk(S) - set of all subsets of S of size k.

Note that communication may be overlapped with computation, so estimating time for a task should allow for this (don’t necessarily sum computation + communication).

User must provide a communication cost for each cell (not provided by a dependency graph).
Assume communication costs are proportional to data volume, ignore startup latency,
(Note that this does not account for how far the data has to travel, only on its volume).
So minimizing communication cost in the algorithm will not maximize locality.  This cost also does not account for congestion.

2.1.4 from multi-objective to constrained mono-objective

Definition 11 (imbalance)
$imb_c(PI_p) = (total - mean) / mean$        for criterion c in part p
$imb(PI) = max imb_c over all PI_p$

Hypergraphs, hyper edges.
Vertices are taken from the original graph.  Hyperedges may span multiple vertices.
One hyper edge for each vertex, contains vertex and its neighbors.  Figure 2.3.1.
A partition of a hypergraph is a partition of its vertices.

$cut_{\lambda-1}(e)$ is the amount of data that must be sent from $e$ to each partner cell.
$\lambda-1$ is the number of partners.

2.4
Definition 17: dual graph of a mesh.

NP-completeness of multi-criteria graph partitioning:
$GP_\gamma$ is multi criteria graph problem.
$GPD_\gamma$ multi criteria graph decision problem (“does there exist a partition…”)
Hyafil and Rivest (1973) showed $GDP_1$ is NP-complete when communication cost is the edge cut function.

2.5 vector-of-numbers partitioning (problem subset sum with multiple criteria)
3 survey of algorithms for vector-of-numbers problem

“Dynamic programming can solve the number partitioning problem is pseudo-polynomial time”.

(Gent and Walsh 1998) $n$ numbers drawn uniformly and randomly from [1,u],
$\kappa = \frac{\log_2(u)}{n}$ is the number of perfectly balanced partitions.
$\kappa << 1$ are underconstrained, most candidate solutions are balanced,
$\kappa >> 1$ are overconstrainted, most candidate solutions are not balanced.

3.2.1 greedy algorithm, heuristic, $O(n \log n)$
Places largest number in the lightest part.

3.2.2 karmarkar-karp heuristic KK
Places largest two numbers into different parts.

3.2.3 dynamic programming (Garey and Johnson 1979)
This algorithm only works for integers and is designed for bi-partitioning.
Builds an array for each possible sum and tracks whether a subset exists.

3.2.4 optimized dynamic programming, Horowitz and sahni (HS) and Schroeppel and Shamir (SS)
Also works on integers only.
Computes exhaustive subsets.

3.2.5 complete greedy algorithm (CGA)
Enumerates all possible partitions using a $k$-ary tree.

3.2.6 complete karmarkar-karp heuristics (CKK)

3.2.7 stochastic search

3.3 comparison of number partitioning algorithms
Table 3.3.1, 3.3.2

Chapter 4 mesh partitioning
Recursive bisection, geometric algorithms, topological (spectral partitioning), 
Greedy graph growing, topological refinement (kernighan-lin 1970, vertex swapping).
Fiduccia-Mattheyes (FM), vertex movement (as opposed to swapping).

4.5 the multilevel algorithm
Coarsening phase: cluster vertices together.
Direct partitioning algorithm applied to coarsened graph.
Expansion phase (uncoarsen).

5. Analysis of fitness landscapes

A lot of unsurprising discussions about the state space and existence of solutions

6. Analysis of new coarsening schemes

Survey of graph partitioning schemes, maybe read this:
Buluç, A, Meyerhenke, H., Safro, I., Sanders, P. and Schulz, C., 2015. Recent advances in graph partitioning. In Algorithm Engineering: Selected Results and Surveys, LNCS 9220. Springer-Verlag. 
This chapter will not introduce any new coarsening scheme.
It analyzes existing schemes.

coarsening is a maximum weight matching problem: coarsen (group together) where the edge weights are maximal, to remove them from the communication pattern.
Exact solutions are too expensive even though they are polynomial in time.

Heavy edge matching, effect of ordering schemes based on edge weights, vertex weights.


7. Initial partitioning phase

Vector-of-numbers partitioning.
Descent methods (greedy steepest descent) chooses to remove a vertex to reduce imbalance.
Lots of details.


8. Local optimization refined reduces communication cost while preserving balance

This is the expansion phase of the multilevel algorithm.
FM: move one vertex at a time;
KL: swap pairs of vertices.
Tie-breaking between moves of the same gain.

Not relaxing the imbalance tolerance allows us to define a simple version of the FM algorithm. The key features follow classic implementations: we do not allow moves that imbalance the partition more than the tolerance, we stop a pass after 120 moves of negative gains, and we stop the algorithm when the last pass did not improve the communication cost. 

Therefore, unlike most partitioning tools, which often return partitions that are not solutions because they prioritize the communication cost over the partition imbalance, we designed our multilevel algorithm to first aim at returning a balanced partition, and then minimize the communication cost. To this extent, we first proposed in Chapter 6 several coarsening schemes and analyzed them, and we will compare their performance in the next part. Then, in Chapter 7, we defined two vector-of-numbers initial partitioning algorithms that focus on minimizing the imbalance of the initial partition. Finally, in Chapter 8, we described our refinement algorithm, which is simple because it focuses plainly on minimizing the communication cost. In the next part, we will examine the consequences of using different coarsening schemes and different initial partitioning algorithms. 

9. Experimental results

3D mesh named LMJ has 484,356 vertices.
Used weight distributions based on particle-in-cell calculations.
MetTiS often failed to return solutions.

10. Comparison of heuristics

The main results are: 
our initial partitioning algorithm VNBest returns partitions of the coarsest graph with the smallest imbalance on average. Nevertheless, the evolution of the average imbalance showed that GGG was able to catch up with VNBest at finer levels; 

for a tolerance of 5\%, using GGG or VNBest does not change much the distribution of the communication cost of the returned solutions. However, when the tolerance tightens, algorithms relying on VNBest do not manage to return as many solutions close to an optimal solution; 

enforcing restrictions on the vertex weights during the coarsening phase leads each initial partitioning algorithm to return more balanced solutions. This result supports our claim that bounding the vertex weights simplifies the search for a solution; 
restrictions also decreased the communication cost of the returned so- lutions, which sides with our claim that bounding the vertex weights simplifies the search for an optimal solution. However, how to find the optimal bound remains an open question. In our experiments, we relied on the Restrict 8 policy; 

ordering the vertices before computing the matching is of great importance. In particular, the order must not always schedule the vertices in the same order, and must take into account the graph topology; 

comparison with other partitioning tools showed that, unlike the others, our approach always returned a solution. In particular, MeTiS policy to relax the imbalance tolerance leads to counter-intuitive results, when MeTiS finds more solutions for a tighter tolerance. This was particularly true on the industrial case, for which MeTiS returned 0 solutions for a tolerance of 5\%, but found some when the tolerance was of 1\%, whereas Crack and Scotch managed to always return a solution; 

in addition to always returning a solution, our approach achieves to return partitions of small communication cost, even clearly smaller in some cases; 

our approach is also efficient for k-partitioning using the recursive bisection scheme. The communication cost is as small as when using other partitioning tools, and on the industrial test case, it manages, unlike MeTiS, to always return a solution. 



\subsection{task based}

\cite{CPE:CPE1631}\\
\cite{Bhatti2017}\\
Locality-aware task scheduling for homogeneous parallel computing systems
- LeTS heuritic (locality of data + load balancing)\\ 
- working task group formation phase\\
	- capture reuse of data across tasks\\
- working task group ordering phase\\
	- minimzes reuse distance of shared data between tasks\\

-NOTE: offline list scheduler takes the app profile (task graph), generates
schedule based on that.\\ 
-IMPORTANT: Locality-aware task management for unstructured parallelism: a quantitative limit study
in SPAA 2013.\\
   ---- this first introduced this notion\\
   ---- 2x over randomized work stealing\\
	---- cache levels + cache private/shared control factor in local
scheduling.\\
	----METIS divides the vertices from a task sharing graph into a given number
of groups, while trying to (a) maximize the sum of edge weights internal to each
group (i.e., data sharing captured by a task group), and (b) equalize the sum of
vertex weights in each group (i.e., balance load).\\
  ----a task group should be formed so that the working set of the group fits in cache.\\
  ----task groups should be formed so that the cache size falls between the
union and sum of task footprint\\
  ----given a shared cache line, the sharing degree denotes the average number
of tasks within a task group that share it.\\
  ----sharing degree indicates the potential impact of task ordering.
Specifically, a high sharing degree implies that data is shared by a large
fraction of tasks.\\
  ----clustered sharing: The sudden increase in cut cost means that the task
group size became small enough that tasks sharing their key data structures have
been separated into different groups. Ordering those task groups so that they
execute consecutively will increase locality.\\
  ----
showcase improvements by varying CCR (compute-comm ration), task graphs and
num
of cores.\\
\cite{5599103}\\

\cite{Posner2018}\\
Hybrid work stealing of locality-flexible and cancelable tasks for the APGAS library\\
- based on lifelines + forkjoin pool (java)\\
- cancellable asyncAny tasks; hmm. \\
- identifies an opportunity of corrected load balancing, is it the same as a task migrating twice\\
- hybrid schemes in general\\
- able to obtain near linear speedup\\

Legion does not have this\\
	- cancellation is useful in search tasks.\\
	- nature of cancellable tasks are that they are independent\\

phases of balance+work\\
 work stealing + proactive + reactive\\

\cite{CCGrid2018}\\
Handling Transient and Persistent Imbalance Together in Distributed and Shared Memory
- transient \\
- persistent \\
- heterogenous clusters intra- and inter-node\\
- chare objects then adaptively share work with other cores in the same process,
  exposing fine-grained tasks only to the extent that otherwise idle cores are
available to help execute them.\\
- A pure task model with randomized work stealing, or a pure dynamic schedule in
  OpenMP, sacrifices locality significantly to an extent that often nullifies
the benefits of dynamic load balancing\\
- we propose is to utilize a relatively infrequent periodic assignment of work
  to cores based on load measure- ment, combined with user assisted creation of
potential tasks from the work assigned to each core that the runtime can choose
to make available to other cores.\\
- PARTITIONING ON DEMAND (KEY LEGION IDEA)\\
- openMP parallelfor task generation\\
- overhead reduced by using principle of persistence (current an indication of
  future)\\
\cite{8025281}\\
\cite{7307597}\\
\cite{Galvez:2017:ATM:3079079.3079104} \\% Charm++


\subsection{Work Stealing}
\cite{Yang2017}
\cite{Chen:2015:LWS:2775085.2766450}
\cite{Blumofe:1999:SMC:324133.324234}
\cite{Cilk}
\cite{Saraswat:2011:LGL:1941553.1941582}
\cite{Mitzenmacher:1998:ALS:277651.277687}

\subsection{other}

\cite{Gao:2017:MPL:3110224.3110240}
\cite{CAMPOS20001213}
\cite{PINAR2004974}
\cite{7551381}
\cite{Menon:2013:DDL:2503210.2503284}
\cite{Liu:2017}
\cite{SEVERIUKHINA2017139}
\cite{7965131}


\subsection{reviews}

\cite{Teresco_2partitioning}

\section{MEETING NOTES}

\subsection{what we are trying to do is expand on alan's equation}
	- roofline analysis\\
	- work stealing \\
	- expand mapping hierarchical \\
	- work-first vs share-first policy, can this even be captured in a formula ? \\
	- adding penalty for a failed steal\\
	- accounting for critical path length\\
	- with workload characterization\\
		- effect of diffusive policies, update lv based on neighbor difference, second-order diffusion, improved diffusion,\\
 		- chemotaxis-inspired, dimension \\

	- what about task selection (i.e., mapping) is it already covered\\

	- graph partitioning, zoltan, parmetis\\
\subsection{Some Thoughts}
- in total, are we extending Diffusive methods to capture workloads without this laplacian angle ?, i.e., where the workload characterization does not follow a  laplacian.. \\

- in which case, do we need a result such as below ?\\
if requests are made randomly by P processors to P deques with each processor allowed at most one outstanding request, then the total amount of time that the processors spend waiting for their requests to be satisfied is likely to be proportional to the total number M of requests, no matter which processors make the re- quests and no matter how the requests are distributed over time\\
We can  view each ball and each bin as being owned by a distinct processor. If a ball is in the reservoir, it means that the ball?s owner is not making a steal request. If a ball is in a bin, it means that the ball?s owner has made a steal request to the deque of the bin?s owner, but that the request has not yet been satisfied. When a ball is removed from a bin and returned to the reservoir, it means that the request has been serviced.\\
Title of the paper: balls, bins and more\\


\subsection{misc}
- the task comm matrix (i.e., a dependence matrix of tasks) can be formed for regular and irregular, hence these techniques apply to both. 
- GPU papers not only do mapping to GPU, but also change the nature of the
  tasks, e..g, how threads in a warp access memory, should this be done by the compiler ? what is the feedback to the load
balancer? 
Instruction divergence
Load balancing between threads of a warp
Memory access conflicts between threads
Cache performance among threads
Local memory utilization 


\subsection{comprehensive model.}
	- transfer of the tasks cost,\\ 
		-in the diffusive models this transfer cost was not considered, but it could be a constant per task\\
		-finding the task to transfer to.\\
		- include the cost of pushing things to GPU, also AOS to SOA. \\
	- critical path, control flow dependency graph\\
		- who should run before whom is not there in the cost model ?\\
		- graph that we have not accounted for, i.e., depenency graph. \\
	- d, i.e., distance matrix between processors could change\\
	- c, i.e., comm matrix between tasks could change, i.e., dimensionality could change\\
	
\paragraph{Nice to have}
	- mapping dependencies, or is it? \\
		- is not accounted for, \\
		- penalty, if e.g., 2 to gpu1, 2 to gpu2.\\
	- change in d and c is it equivalent to evaluating to the cost, i.e., not the same from the last step.\\
	- theoritical proof of the eigen value is good but the cost of the eigen value calculation, i.e., \\ 


\paragraph{Alex Slides}
critique all the  other, taxonomy of the algorithms showing how they relate to the model, the cost model being comprehensive. 
algorithm does not capture the transfer cost. 


\paragraph{move the below to appropriate places in this document}
looking at the mapping papers and tami tamur's paper on rebalancing
------------------------------------------------------------------
Partitioning Sparse Matrices with Eigenvectors of Graphs - A heuristic algorithm is designed to compute a vertex separator in a general graph by first computing an edge separator in the graph from an eigenvector of the Laplacian matrix, and then using a maximum matching in a subgraph to compute the vertex separator. \\

we could describe the comm problem as an equilibrium problem, load balance is another equilibrium, mapping is equilibrium.; consider the nature of the tasks in terms of compute intensive vs comm intensive.\\


\printbibliography

\end{document}
