\documentclass{article}
\usepackage[style=numeric]{biblatex}
\usepackage[margin=1.0in]{geometry}


\addbibresource{biblatex-examples.bib}
\addbibresource{load_balancing_bibliography.bib}

%% declare this to include abstrats in the printed bibliography
%%
\DeclareFieldFormat{abstract}{\par\small#1}
\renewbibmacro*{finentry}{\printfield{abstract}\finentry}


\begin{document}

\title{Load balancing bibliography}
\author{Alan Heirich and Karthik Murthy}

\maketitle

\section{Categories}

\subsection{introduction}

\begin{itemize}
\item
Unbalanced tree search benchmark used by lifeline mapper
\cite{Saraswat:2011:LGL:1941553.1941582}.
Is there a standard benchmark for AMR simulations?

\item
Does the LB algorithm depend on the application?  Does it depend on the programming model (Legion)?  Is it independent of both?

\end{itemize}

\subsection{Category axes}

\begin{itemize}
\item
local vs. global: does the entire computer system have to participate, or can a small subset balance locally?

\item
nearest neighbor vs. long range exchanges

\item 
static vs. dynamic

\item
expensive vs. cheap (related to static vs. dynamic)

\item
embarassingly parallel vs. interconnected: does mapping matter?

\item
continuous domain (mesh simulations) vs. discrete (tree search)

\end{itemize}


\subsection{diffusion}

\let\clearpage\relax
%\begin{center}
%\begin{tabular}{ |c|c| } 
% \hline
% challenges:\\
- rebalance dynamically\\
- balancing based on entities\\
- diffusion with a local-exchange, i.e., bottom-up is bad\\
pros:\\
-log n approach since its top down\\
-communication cost accounted for, in a way\\
cons:\\
-a binary tree approach\\
-assume work load characterization\\
NOTE: Add [5] reference to the collection, its asurvey paper on load balancing\\
 & cell 1\\ 
% cell4 & \include{leiber} \\ 
% cell7 & cell8 \\ 
% \hline
%\end{tabular}
%\end{center}

\cite{Diamond:2017:DLB:3148226.3148236}\\
\cite{HORTON1993209}\\
challenges:\\
- rebalance dynamically\\
- balancing based on entities\\
- diffusion with a local-exchange, i.e., bottom-up is bad\\
pros:\\
-log n approach since its top down\\
-communication cost accounted for, in a way\\
cons:\\
-a binary tree approach\\
-assume work load characterization\\
NOTE: Add [5] reference to the collection, its asurvey paper on load balancing\\

\cite{Deng:2010:HDB:1889863.1889910}\\
\cite{Lieber:2016:PDL:2966884.2966887}\\
challenges:\\
-diffusion vs geometrical vs graph-based methods\\
-tasks are migrated based on geometrical coordinates or graph topolgogy\\
-geometrical good balance but lot of migration, e.g., gossip\\
-diffusion never really considered in HPC\\
this is a survey paper[a very good reference]:\\
- load transfer vector prepared by node to move tasks to another node\\
- CHEBY algorithm not uses difference in work load\\
- PARMETIS uses diffusion, ZOLTAN library and GLB library should be in mind\\
- chemotaxis uses capacity on me - load on target\\
- which tasks to send based on comm reduction, i.e., comm with neighbors only\\
their questions are our questions:\\
-how can we apply diffusion on today’s common hardware topologies like fat trees?\\
-What is a fast and high-quality method for task selection allowing to trade-off balance, migration and edge-cut? \\
-How do we scalably implement the termination criterion for diffusion when no fast collectives are available? E. g. use approx or auto-tune iteration count?\\ 

\cite{ROTARU2004481}\\
\cite{ParabolicLB}\\
\cite{CYBENKO1989279}\\
\cite{10.2307/2584287}\\
\cite{Boillat:1990:LBP:95324.95326}\\
\cite{XU199572}
challenges:\\
-mapping + load balancing at the same time\\
-a form of diffusion is the answer for both\\
-goals of mapping, workload balance and comm time\\
pros:\\
-based on the laplacian matrix of communicating processes\\ 
-delay and fault tolerant, insensitive to problem scale, convergence depends on initial conditions (hmm)\\
cons:\\
-np-complete problem\\
-m to be 2, planar connection networks, but arbitrary topologies like hierarchical, maybe linearization, not sure: did not understand the mapping to torus by a broadcast-type strategy, what about binomial dissemination?.\\
-reduction of the problem for a specific instance ? isnt it ?



\medskip
notes on \cite{Diamond:2017:DLB:3148226.3148236}
from Alan

finite element, finite volume unstructured adapting meshes

Diffusive partition improvement, application specified criteria

N-graph: hyper graph structure represents relations among elements

Diffusion is performed on the N-graph

A multigraph allows multiple edges between a pair of vertices

The N-graph does this for hyper graphs

Saves on memory versus just a graph

Imbalance = $T_max / T_mean$

Does not explain how to compute the transfer amount

Graph distance: migrate elements in order according to their distance from the cell center

Experiments : billion element mesh airplane tail structure

Argonne Mira blue gene Q

$128*2^{10}$ to $512*2^{10}$ elements cases

Showed reasonable improvement, 1.5 imbalance to 1.12

Did worse on larger problems


\medskip

notes on \cite{HORTON1993209}
from Alan

fea unstructured adaptive mesh

No topology assumptions

Multilevel algorithm complexity is logarithmic in number of processors

Diffusion methods may require many iterations, see Boillat claims of $O(n^2)$ iterations on n processors

Claim: pairwise diffusion can result in large load imbalance (I don’t think this is true of Laplace iteration although low frequency disturbances subside slowly)
See Cybenko claim of $log_2(n)$ steps iteration, uses hypercube topology

The algorithm here achieves $O(log_2 n)$ but does not depend on topology

Local communication costs less than nonlocal : hypercubes (this will always be true, just how much)

Load balancer should respect existing adjacency relationships of the domain

topology of the mesh may not match topology of computers

Basic diffusion method pairwise exchange of $0.5*(l_i - l_j)$ units of work

Multilevel algorithm aims to eliminate large scale imbalances

At each level divide processors into two sets and balance them as with two individuals

No explanation of how to choose these sets

Proof that it takes $log_2(n)$ steps — duh

requires entire system rebalance at once, not a local method

Claims standard diffusion techniques are bad because they don’t guarantee number of iterations


\medskip

notes on
\cite{Deng:2010:HDB:1889863.1889910}
from Alan

Efficient cell selection scheme

Local and global diffusion schemes, global performs best

Global means knows all servers, local means only knows nearest neighbors

Note: reference ou and ranka 1997 solve lb problem as linear programming

Distributed virtual environments prefer fast solution over optimal solution

Experiment using simulated workloads, virtual environment users moving through the environment, environment is partitioned into regions, one region per server


\medskip

notes on
\cite{Lieber:2016:PDL:2966884.2966887}
from Alan


Good survey paper, worth reading again

Compare diffusion to geometric and graph based methods on thousands of nodes

Space filling curves, recursive bisection, parMetis, hierarchical space filling curves

Concludes diffusion has advantages

	.	The second-order algo- rithm [15] extends OD such that the previous iteration’s transfer influences the current. The parameter β ∈ ( 0,2) controls the influence. Optimal values are derived in [9] 

	.	[9]  R. Elsa ̈sser, B. Monien, and R. Preis. Diffusion Schemes for Load Balancing on Heterogeneous Networks. Theory Comput. Sys., 35(3):305–320, 2002. 

	.	[15]  S. Muthukrishnan, B. Ghosh, and M. H. Schultz. First 
and Second Order Diffusive Methods for Rapid, Coarse, Distributed Load Balancing. Theory Comput. Sys., 31:331–354, 1998. 

Survey based on improving diffusion

Original diffusion

Second order diffusion

Improved diffusion called cheby

Chemotaxis-inspired diffusion, additional round of exchange of capacity of the target node guides the diffusion locally

Dimension exchange - Xu and Lau



\medskip

notes on
\cite{ROTARU2004481}
from Alan


Our contribution  can be summarized as follows: we give a direct explicit expression of the balancing flow generated by a generalized diffusion algorithm and we show that this flow has an interesting property, that it is a scaled projection of any other balancing flow in the same heterogeneous environment. We give estimations for the second largest eigenvalue of a generalized diffusion matrix and we estimate the complexity of the proposed algorithm.  We further show that this algorithm has a better convergence factor than the hydrodynamic algorithm [17,18]. Compared to other approaches, the one we consider here offers the advantage of not using parameters that are dependent upon the eigenvalues of the Laplacian of the communication graph. 

Solves load balancing and mapping

Since communication changes so frequently cannot afford to compute Laplacian eigenvalues

Analogy to markov chains

Connections between generalized diffusion matrices and Laplacian spectrum of the graph

Bounds on eigenvalues

Migration flow - expression for the flow

Good paper lots of analysis


Estimations were given for the maximum number of steps that such an iterative process may take to balance

purpose, some general bounds were formulated for the second largest eigenvalue of a generalized diffusion matrix. These bounds were also used to show that there are generalized diffusion algorithms that theoretically converge faster than the hydrodynamic algorithm 



\medskip

notes on
\cite{10.2307/2584287}
from Alan


uses successive over relaxation to find an optimal step size for convergence

I wonder: do these convergence issues really matter?  Is the simplest scheme good enough?


\medskip

notes on
\cite{Boillat:1990:LBP:95324.95326}
from Alan

Show polynomial time convergence to equilibrium

Remark 5. In the discrete case, i.e. working with individual processes, our problem is equivalent to the random walk problem in graphs[20] 
20. D. Aldous. ‘An inaoduction to covering problems for random walks on graphs’.J. Theoretical 
Probability, 2(1), 87-89 (1989).






\subsection{game theory}

\cite{GROSU20051022}
\cite{doi:10.1142/S0219198902000574}
\cite{7967109}
\cite{BELIKOVETSKY201616}



\medskip

notes on
\cite{GROSU20051022}
from Alan


Static load balancing problem

Noncooperative game: processors work independently to arrive at equilibrium

Characterize Nash equilibrium and derive greedy algorithm to compute it

Assume Poisson arrivals, exponentially distributed task times

$\phi_i$ job generating rate at node I

$s_{ji}$ fraction of user j tasks to be sent to node i

$\mu_i$ processing rate at node i 

Load balancing strategy for user j is a vector of ${ s_{ji} }$

Minimize response time for user j

Remark: this is for multiple users (humans) submitting jobs to a distributed system (cluster).

Our case is one user (application) submitting tasks to an exascale system

“Best reply” for a user is a strategy that gives minimal response time for that user in light of other users strategies

Similar problem for one user treated in Tang and Chanson[35]
X. Tang, S.T. Chanson, Optimizing static job scheduling in a network
of heterogeneous computers, in: Proceedings of the International
Conference on Parallel Processing, August 2000, 373–382. Not very interesting.

Equation (8) defines the BEST\_REPLY solution

Execution time is O(n log n) due to the need to sort computers by workload

Otherwise it would be O(n)

This algorithm requires global knowledge of the workload of every computer at every agent and knowledge of all users strategies

Not scalable

Compared to two other lb schemes, one does a global optimization based on global knowledge, and an IOS scheme that gives good quality results

Experiments on 16 node cluster



\medskip

notes on
\cite{doi:10.1142/S0219198902000574}
from Alan


Establish uniqueness of Nash equilibria

No software experiment

Not relevant


\medskip


notes on
\cite{7967109}
from Alan

Entirely theoretical result, random movement of tasks with uniform weights




\medskip

notes on
\cite{BELIKOVETSKY201616}
from alan

Load Rebalancing Games in Dynamic Systems with Migration Costs 
Initial assignment of jobs to identical parallel machines

Machines are added or subtracted

Extension parameter $\sigma$ is added to the cost of a job after it is moved (This does not seem right, movement should incur a one time cost, not a repeated cost)

Paper proves existence and calculation of Nash equilibrium and Strong Nash equilibrium

Under some assumptions any stable modified schedule approximates well an optimal schedule

“Each job incurs a cost which is equal to the total load on the machine it is assigned to” … ????
See game theoretic treatments of this problem 12, 2, 5, 8, survey in 17
	.	[2]  N. Andelman, M. Feldman, and Y. Mansour. Strong Price of Anarchy. In SODA, 2007 
	.	[5]  A. Czumaj and B. V ̈ocking. Tight bounds for worst-case equilibria. In ACM Transactions on Algo- 
rithms, vol.3(1), 2007 
	.	[8]  A. Fiat, H. Kaplan, M. Levi, and S. Olonetsky. Strong Price of Anarchy for Machine Load Balancing. In ICALP, 2007. 
	.	[12] R.L. Graham. Bounds on Multiprocessing Timing Anomalies. SIAM J. Appl. Math., 17:263–269, 1969. 
	.	[17] B. V ̈ocking. In N. Nisan, T. Roughgarden, E. Tardos and V. Vazirani, eds., Algorithmic Game Theory. Chapter 20: Selfish Load Balancing. Cambridge University Press, 2007. 
$s_0$ assignment of $n$ jobs on $m_0$ machines
$m’$ machine are added or removed
Seek a Nash equilibrium $s$ where no machine can improve its situation by changing machines
``Assume that some preprocessing is done at the time a client is assigned to a server, before the download actually begins (e.g., locating the required file, format conversion, etc.). Clients might choose to switch to a mirror server. Such a change would require repeating the preprocessing work on the new server.   Another example of a system in which extension penalty occurs is of an RPC (Remote Proce- dure Call) service. In this service, a cloud of servers enables service to simultaneous users. When the system is upgraded, more virtual servers are added. Users might switch to the new servers and get a better service (with less congestion), however, some set-up time and configuration tuning is required for each new user.  Note that in all the above applications, the delay caused due to a migration is independent of the migrating job. 
’’
These sound like one-time costs to me hence my objection to this analysis, also the cost may depend on the job (amount of data movement, size of code, number of open files, etc)

Section 1.1 some issues with notation, use of $L_i(s)$ and $L_s(j)$ to mean different things?
“Price of anarchy” = ratio between max cost of a Nash equilibrium and the optimum schedule (load imbalance)
“Price of stability” = ratio between min cost of a Nash equilibrium and the optimum schedule
A set of players form a “coalition” if each job moves and strictly reduces its cost
Assignment is a “strong Nash equilibrium” if there exists no such coalition
Similar notions of strong price of anarchy, strong price of stability

``The simple greedy List-scheduling (LS) algorithm [11] provides a $(2 - 1/m)$approximation to the minimum makespan problem. A bit better approximation ratio of  $((4/3) - (1/3m))$ is guaranteed by the Longest Processing Time (LPT) algorithm [12]. A PTAS for the minimum makespan problem on identical machines is given in [13]. ‘’

``We show that any job scheduling game with added or removed machines possesses at least one Nash equilibrium schedule. Moreover, some optimal solution is also a Nash equilibrium, and thus, the price of stability is 1. We show that in general, the price of anarchy is unbounded when machines are either added or removed. ‘’

``We note that in a dynamic setting in which machines are added or removed and migrations 
are free of cost (i.e., when $\sigma$ = 0), then the results known for classic load balancing games apply.   In particular, the P oA assuming $\sigma$ = 0 is $2 - (2 / (m+1))$ for a game with m machines in the modified $m+1$  systems. The proofs are identical to the proofs for a fixed number of machines. Thus, the difference between our results and the results for the classic load balancing game are due to the migration penalty. ‘’

Sections on machine addition, machine removed, coalitions.  Each case proves existence of solutions.






\subsection{mapping}

\cite{doi:10.1137/0611030}
\cite{4227986}
\cite{doi:10.1142/S0129054197000215}
\cite{Sbirlea:2014:BMS:2628071.2628090}

\subsection{heterogeneous}

\cite{Flegar:2017:OLI:3149704.3149767}
\cite{8082085}
\cite{7993387}
\cite{Cederman:2008:DLB:1413957.1413967}
\cite{10.1007/978-981-10-6442-5_56}
\cite{dlbgraphgpu}




\subsection{large scale}

\cite{PEARCE2017}
\cite{BERLINSKA201814}
\cite{8017633}
\cite{DEVINE2005133}
\cite{javataskpool}
\cite{barat:tel-01672546}



\subsection{task based}

\cite{CPE:CPE1631}
\cite{Bhatti2017}
\cite{5599103}
\cite{Posner2018}
\cite{CCGrid2018}
\cite{8025281}
\cite{7307597}
\cite{Galvez:2017:ATM:3079079.3079104} % Charm++


\subsection{Work Stealing}
\cite{Yang2017}
\cite{Chen:2015:LWS:2775085.2766450}
\cite{Blumofe:1999:SMC:324133.324234}
\cite{Cilk}
\cite{Saraswat:2011:LGL:1941553.1941582}

\subsection{other}

\cite{Gao:2017:MPL:3110224.3110240}
\cite{CAMPOS20001213}
\cite{PINAR2004974}
\cite{7551381}
\cite{Menon:2013:DDL:2503210.2503284}
\cite{Liu:2017}
\cite{SEVERIUKHINA2017139}
\cite{7965131}


\subsection{reviews}

\cite{Teresco_2partitioning}

\printbibliography

\end{document}
